\documentclass[10pt,twocolumn,letterpaper]{article}
\usepackage{cvpr}
\usepackage{times}
\usepackage{epsfig}
\usepackage{graphicx}
\usepackage{caption}
\usepackage{float}
\usepackage{subcaption}
\usepackage{amsmath}
\usepackage{amssymb}
\usepackage{algorithmic,algorithm}
\newcommand{\maximize}[1]{\underset{#1}{\operatorname{maximize}\;}}
\newcommand{\minimize}[1]{\underset{#1}{\operatorname{minimize}\;}}
\newcommand{\argmin}[1]{\underset{#1}{\operatorname{arg}\,\operatorname{min}}\;}
% Include other packages here, before hyperref.

% If you comment hyperref and then uncomment it, you should delete
% egpaper.aux before re-running latex.  (Or just hit 'q' on the first latex
% run, let it finish, and you should be clear).
%\usepackage[pagebackref=true,breaklinks=true,letterpaper=true,colorlinks,bookmarks=false]{hyperref}

\cvprfinalcopy % *** Uncomment this line for the final submission

\def\cvprPaperID{****} % *** Enter the CVPR Paper ID here
\def\httilde{\mbox{\tt\raisebox{-.5ex}{\symbol{126}}}}

% Pages are numbered in submission mode, and unnumbered in camera-ready
\ifcvprfinal\pagestyle{empty}\fi
\begin{document}

%%%%%%%%% TITLE
\title{CS229 Project Report \\
Automated Stock Trading Using Machine Learning Algorithms}

\author{Tianxin Dai\\
{\tt\small tianxind@stanford.edu}
% For a paper whose authors are all at the same institution,
% omit the following lines up until the closing ``}''.
% Additional authors and addresses can be added with ``\and'',
% just like the second author.
% To save space, use either the email address or home page, not both
\and
Arpan Shah\\
{\tt\small ashah29@stanford.edu}
\and
Hongxia Zhong\\
{\tt\small hongxia.zhong@stanford.edu}
}

\maketitle
\thispagestyle{empty}

\section{Introduction}
The use of algorithms to make trading decisions has become a prevalent practice in major stock exchanges
of the world. Algorithmic trading, sometimes called high-frequency trading, is
the use of automated systems to identify true signals among massive amounts of data that capture the underlying stock market dynamics. Machine Learning has therefore been central to the process of algorithmic trading because it provides powerful tools to extract patterns from the seemingly chaotic market trends. This project, in particular, learns models from Bloomberg data to predict stock price changes and aims to make profit over time.

Our algorithm tries to make a profit by predicting the price trend in the next minute based on the information seen earlier on that
day. We make a buying decision if we predict an uptrend with a high confidence.  

\section{Data set}
We use data obtained from Bloomberg terminal.
Talk about how we bucketize it into minute-by-minute data (high/low/open/close) why we make this decision.

Also talk about universe selection.

Give an example with mean-reversion.

\section{Approach}
A brief paragraph about two approaches, pros/cons of each.

\subsection{Logistic regression}
We first used logistic regression to relate the percentage price change in the current minute to percentage price and volume
change in the last minute. The intuition is quite simple: If the price went up in the last minute and volume also went up,
that means more people are buying in the stock, and the uptrend should continue. Similarly, if the price went down in the 
last minute and volume went up, that means more people are selling the stock, and the downtrend should continue. On
the other hand, if we are seeing a decrease in volume, the previous trend should revert. For our first attempt, we used
six features to represent the trend in price and volume. Suppose we are trying to predict percentage change in closing price
from $t - 1$ to $t$, our regression variable $y$ would be
\begin{equation}
\left(P_{close}^{(t)} - P_{close}^{(t-1)}\right)/P_{close}^{(t-1)}
\end{equation}
Our six features would be:
\begin{enumerate}
\item $\left(P_{open}^{(t)} - P_{open}^{(t-1)}\right)/P_{open}^{(t-1)}$
\item $\left(P_{high}^{(t)} - P_{high}^{(t-1)}\right)/P_{high}^{(t-1)}$
\item $\left(P_{low}^{(t)} - P_{low}^{(t-1)}\right)/P_{low}^{(t-1)}$
\item $\left(V_{open}^{(t)} - V_{open}^{(t-1)}\right)/V_{open}^{(t-1)}$
\item $\left(V_{high}^{(t)} - V_{high}^{(t-1)}\right)/V_{high}^{(t-1)}$
\item $\left(V_{low}^{(t)} - V_{low}^{(t-1)}\right)/V_{low}^{(t-1)}$
\end{enumerate} 
However, we could not find a logistic model that fits this set of high-dimensional features well. We found out that our
high-dimensional features are very noisy, so it is hard to fit a line through the data points. 

In an attempt to reduce the dimensionality of our regression features, we selected the two most significant features: 
\begin{enumerate}
\item $\left(P_{open}^{(t)} - P_{open}^{(t-1)}\right)/P_{open}^{(t-1)}$
\item $\left(V_{open}^{(t)} - V_{open}^{(t-1)}\right)/V_{open}^{(t-1)}$
\end{enumerate}

\[\frac{\%\Delta P_{open}^t}{5MinHighLowSpread}\]
While logistic regression is able to fit a model with the two-dimensional feature set, the result is quite poor. The average profit made via trading on the selected 25 stocks is -? with a x\% precision and y\% recall. Then we tested our logistic regression model on the training set (9:30am to 11:00am) again and the results are also quite poor (give stats), this suggests that our model is suffering from high variance, and we must choose more descriptive features while keeping low dimensionality. Specifically, the percentage changes in price and volume of the one-minute timeframes are extremely volatile quantities, they are influenced by general trends of the market which normally lasts longer than one minute. For example, if the market enters a mode of frequent fluctuations in a certain time, then large 1-minute percentage changes in price or volume become less significant in comparison to the general trend. On the other hand, in a smoother market, even small percentage changes need to be considered with higher weights.

As a result, we decide to model the local general trend of the market from the high-low spread of the previous 5 minutes of any given time. Instead of using one-minute percentage changes, we choose to use percentage change over 5 minute high-low spread ratio of price and volume as our features. (maybe use math notations to express). Then we retrain the logistic regression model with the new features and we achieved quite promising results (x profit, y\% precision, z\% recall).

\subsection{Conditional Random Fields and Graphcuts}
Conditional random fields (CRFs) are the probabilistic model we use for segmention, which defines a conditional probability distribution of label sequences given the observation sequences. Suppose $x$ is the features of the image, $y \in \{-1, +1\}$ is the segmentation label, with $-1$ denoting background and $+1$ denoting foreground. The CRF can be formulated as
\begin{equation}
P(y\mid x) = \frac{P(y,x)}{\sum_{y'}P(y',x)}  = \frac{1}{Z(x)}\exp(-E(y,x)) 
\end{equation}
where $Z(x)$ is the normalizing function, and $E(y,x)$ is the energy function
computed from various features. $E$ consists of both node potentials and
edge potentials. Intuitively, node potentials encodes the energy of a depth point taking
on a particular label, and edge potential encodes the energy of a pair of depth points
sharing the same label. In particular, $E$ can be written as\cite{AlexRGB}
\begin{equation}
E(y,x) = \sum_{i\in \Phi_v} w_i \sum_{j \in v_i} \phi_j^{(i)}(y,x)
+ \sum_{i\in \Phi_{\mathcal{E}}} w_i \sum_{{(j,k)}\in \mathcal{N}_i} \phi_{jk}^{(i)}(y,x)
\end{equation}
Here, $\Phi_v$ is the set of node potential indices (i.e. one for each type of node
potential such as distance to previous foreground or bilateral filter), $v_i$ is the
set of all node indices for this potential type (one per depth point), and 
$\phi_j^{(i)}$ is the node potential of type $i$ at depth point $j$. Similarly,
$\Phi_\mathcal{E}$ is the set of edge potential indices, $\mathcal{N}_i$ is the 
edge set for edge potential type $i$. In our implementation, we draw an edge between depth point $j$ and each depth point $k$ within a hand-picked radius of point $j$. $\phi_{jk}^{(i)}$ is the edge potential between depth point $j$ and $k$
for edge potentials of type $i$. The weight $w_i$ describes how important the
$i$th potential is.

The goal of MAP inference in this model is to choose the most likely segmentation $y$ given the features $x$ by solving\cite{AlexRGB}
\begin{equation}
\maximize{y} P(y \mid x) = \minimize{y} E(y,x)
\end{equation}

The solution of this problem can be efficiently computed using graph cuts.
We construct graph $G$ with all depth points as our node set, and edges between each depth point and its neighbors as our edge set. Then we assign
node potential and edge potential using the definition of $E$ above. We start
with the seed frame, run graph cuts every frame in the sequence and use the current frame as ground truth for segmenting the next frame. Specifically,
\begin{algorithm}[H]
\caption{Graphcuts segmentation. } \label{gc_seg}
\begin{algorithmic}[1]
\STATE Let $D = \{d_1, d_2,\cdots,d_k\}$ be the $k$-frame-sequence we want to segment, where $d_i$ is the $i$th RGBD frame. Let $Y = \{y_1\}$ be the segmentation of this
sequence. Initially, we only have $y_1$ which is the hand-segmented ground truth for the seed frame. 
\FOR{$i = 2, 3,\cdots,k$}{
\STATE Construct graph $G$ with nodes $V = \{v_1, v_2, \cdots, v_n\}$ where $v_i$'s are the depth points of $d_i$.
\STATE Use $y_{i-1}$ as ground truth labeling when computing node and edge
potentials. 
\FOR{$v_j \in V$}{
\STATE $\phi_{j_{source}} = \sum_{i\in\Phi_v} w_i \phi_{j_{source}}^{(i)}$ and $\phi_{j_{sink}} = \sum_{i\in\Phi_v} w_i \phi_{j_{sink}}^{(i)}$ where 
$\phi_{j_{source}}^{(i)}$ is the source potential of type $i$ at depth point $v_j$, $\phi_{j_{sink}}^{(i)}$ is the corresponding sink potential, and $w_i$ is the weight
for node potential type $i$.
A large positive
source potential means that $v_j$ is more likely to be foreground and a large
positive sink potential means that $v_j$ is more likely to be background.  
\STATE Add an edge $e_{jk}$ for every depth point $v_k$ that is within radius
$r$ of $v_j$.
\FOR{each $e_{jk}$}{
\STATE $\phi_{jk} = \sum_{i\in\Phi_\mathcal{E}} w_i \phi_{jk}^{(i)}$ where
$\phi_{jk}^{(i)}$ is the edge potential type $i$ for $e_{jk}$ and $w_i$ is
the weight for edge potential type $i$. A large edge potential
means that points $j$ and $k$ are more likely to share the same label.  
}\ENDFOR
}\ENDFOR
\STATE Run maxflow on $G$. Let $y_i$ be the source and sink labels produced
by maxflow, where source denotes foreground and sink denotes background.
\STATE $Y = Y \cup y_i$. 
}\ENDFOR
\end{algorithmic}
\end{algorithm}

\subsection{Energy function terms}
We now discuss different node and edge potentials we use in our graph cuts algorithm.
\subsubsection{Node potentials} All node potentials have two parts, source
potential and sink potential. We constrain them to $[0, 1]$. We visualize
node potentials to give us an intuitive sense how they help the performance
of our algorithm. Sample visualizations can be found in Figure 1.
\\

\indent \textbf{Distance to previous foreground} This potential is inversely related to the distance of a depth point to its nearest neighbor in the
foreground of the previous frame. Formally,
\begin{equation}
\phi_{i_{source}} = \exp\left(-\frac{\|v_i-v'_k\|_2}{\sigma}\right)
\end{equation}
and
\begin{equation}
\phi_{i_{sink}} = 1 - \phi_{i_{source}}
\end{equation}
where $v'_k$ is $v_i$'s closest neighbor in the foreground of the previous frame,
and $\sigma$ is a hand-picked parameter to control how fast the potential falls off
with distance.

\indent \textbf{Bilateral filter}
To compute bilateral potential of depth point $v_i$, we find all depth points within radius $r$ of $v_i$ in the previous frame. The bilateral potential is a sum of their labels weighted by their distance to $v_i$. Formally,
\begin{equation}
\psi_i = \sum_{v'_k\in N} y'_k \exp\left(-\frac{\|v_i - v'_k\|_2}{\sigma}\right)
\end{equation}
where $N$ is the set of all neighboring points of $v_i$ in previous frame,
$y'_k\in\{-1, +1\}$ is the label of $v'_k$ in previous frame, $\sigma$ is a
hand-picked parameter to control how fast the weights fall off with distance.
And we need to convert $\psi_i$ to sink and source potential in range $[0,1]$.
We can do this by first converting it to $\phi_i \in [-1, +1]$
\begin{equation}
\phi_i = \frac{2}{1+\exp(-\psi_i)}-1
\end{equation}
Then we can convert it to range $[0, 1]$ by
\begin{equation}
\phi_{i_{source}} = \left\{\begin{array}{l r}
\phi_i & \text{if $\phi_i > 0$} \\
0 & \text{if $\phi_i < 0$} \\
\end{array} \right.
\end{equation}
Similarly,
\begin{equation}
\phi_{i_{sink}} = \left\{\begin{array}{l r}
-\phi_i & \text{if $\phi_i < 0$} \\
0 & \text{if $\phi_i  > 0$} \\
\end{array} \right.
\end{equation}

\subsection{Edge potentials}
Edge potentials encodes the energy of two points sharing the same label, i.e.
two points are more likely to share the same label if their edge potential is
big and vice versa. Similar to node potentials, there is a $\sigma$ term in all
edge potentials which controls how fast the potential falls off.
\\

\indent\textbf{3D distance} This potential is inversely related to the 3D
distance between 2 depth points. Formally,
\begin{equation}
\phi_{ij} = \exp\left(-\frac{\|v_i - v_j\|_2}{\sigma}\right)
\end{equation}

\indent\textbf{Color distance} This potential is inversely related to the distance
between the RGB-color of the corresponding pixels of two depth points.
Formally,
\begin{equation}
\phi_{ij} = \exp\left(-\frac{\|c_i - c_j\|_2}{\sigma}\right)
\end{equation}

\indent\textbf{Surface Normal} This potential measures how much the surface
normal changes from one point to another. It is inversely related to the 
angle between surface normals at two points. Formally,
\begin{equation}
\phi_{ij} = \exp\left(-\frac{\arccos(n_i\cdot n_j)}{\sigma}\right)
\end{equation}
where $n_i$ and $n_j$ are the unit-length surface normal vector at depth points
$v_i$ and $v_j$.

\subsection{Learning}
In our algorithm, we use Structural Support Vector Machine (SSVM) \cite{SSVM20, SSVM21} to choose weight $w$ that maximizes the MAP. 
Each training example adds a constraint to our model. SSVM uses an iterative
process to add more and more constraints and computes the weights that maximize the margin between the ground truth labels and the incorrect labels. To simplify the problem, we use the greedy approximation known as 1-slack formuation \cite{SSVM9} . Thus we can formulate the approximation of the margin maximizing problem as follows:
\begin{equation}
\begin{array}{l l}
\minimize{w,\xi} & \frac{1}{2}||w||^2 + C\xi \nonumber \\
\text{subject to} & \xi \ge 0 \\
& \frac{1}{M}\sum_{m=1}^{M}E(\hat{y_m},x_m)-E(y_m, x_m) \\
& \ge \frac{1}{M}\sum_{m=1}^{M}\Delta(y_m, \hat{y_m}) - \xi \nonumber \\
& \forall (\hat{y_1},...,\hat{y_M}) \in \mathcal{Y} \nonumber \\
\end{array}
\end{equation}
where $\Delta$ is a loss function and $\Delta(y_m, y)$ is 0-1 loss. $C$ is a constant while $\xi_m$ is a slack variable for training example $m$, and $\mathcal{Y}$ is the set of possible labelings of an image. Since graph cuts is based on non-negative edge weights, we constrain the edge weights to be non-negative.
\begin{algorithm}
\caption{Structural SVM for tracking. } \label{algo_ssvm}
\begin{algorithmic}[1]
\STATE $\mathcal{D}$ is a set of training examples $(y,x)$ created by
human-labeling.
\STATE $C$ and $\varepsilon$ are constants, chosen by cross validation.
\STATE
\STATE $\mathcal{W} \gets 0$
\REPEAT
\STATE Update the parameters $w$ to maximize the margin.
\begin{equation}
\begin{array}{ll}
\minimize{w,\xi} & \frac{1}{2}||w||^2 + C\xi \\
\text{subject to} & w \ge 0, \xi \ge 0 \\
& \frac{1}{M}\sum_{m=1}^{M}E(\hat{y_m},x_m)-E(y_m, x_m) \\
& \ge \frac{1}{M}\sum_{m=1}^{M}\Delta(y_m, \hat{y_m}) - \xi \\
& \forall (\hat{y_1},...,\hat{y_M}) \in \mathcal{Y} \\
\end{array}
\end{equation}
\FOR { $(y_m, x_m) \in \mathcal{D}$ }
\STATE Find the MAP assignment using graph cuts.
\STATE $\hat{y_m} \gets \argmin{y}E(y,x_m)$,
\ENDFOR
\STATE $\mathcal{W} \gets \mathcal{W} \cup \{\hat{y_1}\, ...,\hat{y_M}\} $
\UNTIL {\begin{equation}
\begin{array}{l}
\frac{1}{M}\sum_{m=1}^{M}\Delta(y_m, \hat{y_m}) - E(\hat{y_m},x_m)\\
-E(y_m, x_m) \le \xi+\varepsilon
\end{array}
\end{equation}
}
\end{algorithmic}
\end{algorithm}

\section{Experiment}
We test our algorithm at different stages of implementation on XXX dataset 
and evaluated our results both qualitatively and quantitatively.

\subsection{Data set}
We use a dataset in the same format as Stanford Track Collection with 6 sequences and 227 frames extracted from natural, unstaged street scenes
with a dense LIDAR system mounted on a car. Each sequence has 2 to 4 tracked objects. Data was recorded while driving and while parked at busy intersections with many people, bicyclists, and cars.

\subsection{Evaluation Metrics}
We evaluate the results both qualitatively and quantitatively. For our camshift baseline, we mainly evaluated it qualitatively by eyeballing the segmentation, while for our graphcuts algorithm, we mainly evaluated it quantitatively by calculating "Normalized Accuracy"\cite{AlexRGB} compared to our hand-segmented ground truth.

Normalized Accuracy is a measure to capture how many points we are classifying correctl normalized by the actual foreground size. Formally,
\begin{equation}
NA = 1-\min\left(1, \frac{N_{fp} + N_{fn}}{N_{foreground}}\right)
\end{equation}
where $N_{fp}$ is the number of false positives (labeled as foreground by our
algorithm but are actually background), $N_{fn}$ is the number of false
negatives (labeled as background by our algorithm but are actually foreground).
And $N_{foreground}$ is the number of foreground points in ground-truth
segmentation.

\section{Experiment}
\subsection{Camshift baseline}
Our baseline does a decent job on the first few frames, but has very high
error after that.
However, it has three major disadvantages. First, Camshift assumes that
the camera does not shift position vastly. If the camera changes position and 
orientation a lot, Camshift has very high error tracking the object.
Second, Camshift computes the position of the tracked object in the new frame
by finding a window that has the highest back projection with regard to the
color histogram of the object in the seed frame. It depends greatly on color,
which is brittle to illumination and photometric variances, etc. Also when the
object's color is close to background, Camshift cannot distinguish between
them.  
Finally, Camshift is a bounding box tracker, while a lot of real world objects
cannot be represented as a box. We cannot apply our Normalized Accuracy
measure to Camshift results because the bounding box gives too many
false positives.
Figure 3 shows sample frames of Camshift segmentation on sequence 1. 

\subsection{Our algorithm}
Compared to Camshift, our algorithm performs significantly better. The final
normalized accuracy on 6 test sequences are shown in the following chart:

Sample frames extracted from 6 sequences are shown in Figure 4.

We tested our algorithm at different stages of implementation. Here is a summary of what features we added at each stage:

\indent\textbf{Stage 1} Used only distance to previous foreground as node potential and 
3D distance as edge potential. Used coordinate system relative to camera on
car.\\
\indent\textbf{Stage 2} Changed coordinate system to smooth coordinates. This is an
absolute coordinate system based on GPS. Therefore, static objects, such as
traffic signs, will have the same coordinates in every frame.\\
\indent\textbf{Stage 3} Added bilateral node potential.\\
\indent\textbf{Stage 4} Added color edge potential.\\
\indent\textbf{Stage 5} Added surface normal edge potential.\\
\indent\textbf{Stage 6} Implemented SSVM to learn weights of different potentials.


\begin{table}[H]
\centering
\caption{Normalized accuracy on all frame at different stages of implementation}
\begin{tabular}{ c|c|c|c|c|c }
Stage 1    &   Stage 2 & Stage 3 &  Stage 4 & Stage 5 & Stage 6 \\
\hline
\hline
0.1393     &     0.3558  &    0.4036   & 0.4139     &  0.4156       &  0.5481\\
0     &     0.0733    &    0.0941   & 0.0778     &  0.076         &  0.2917\\
0     &     0.5770  &    0.5511   & 0.5193     &  0.5295       &  0.6264\\
0.1943    &     0.7512  &    0.7521   & 0.7573     &  0.7641       &   0.7685 \\
0.1943     &     0.7015  &    0.7143   & 0.6917     &  0.7090	    &  0.7300\\
0.1059     &     0.6200  &    0.6584   & 0.6184     &  0.6161       &  0.6867 \\
\hline
\end{tabular}
\end{table}
We can see that [feature] and [feature] give us biggest improvement on
performance. This corresponds with the weights we learned from
SSVM. [feature1] and [feature2] have bigger weights while features such
as color edge potential has a weight close to 0.

\section{Conclusion}
We have applied successfully a segmentation and tracking model for hand-held
camera data to RGBD data captured by autonomous car.
By trial and error we have gained a deep understanding of the model, 
the effects of different features, and SSVM setup. Due to time limit of a class
project, we did not have time to experiment with more features that are more
specific to the nature of autonomous car data. For example,
we cannot calculate surface normal for a lot of far away depth points, because
they do not have neighbors with which we can estimate the plane. If we had
time, we could 
develop a new feature that computes the tangent through a few nearby points
in the same depth ring to replace surface normals. Also, adding surface normal alone still does not get rid of all the points on the ground that touch the tracked objects. We may solve this problem by first applying ground subtraction, and then run our segmentation pipeline.
 However, the current result is satisfying both qualitatively and quantitatively.
 We believe the application of an improved version of the algorithm can help
 autonomous cars be more knowledgeable about the different kinds of obstacles 
 and make finer-grained decisions in all kinds of traffic situations.

%-------------------------------------------------------------------------

{\small
\bibliographystyle{ieee}

\begin{thebibliography}{20}
\bibitem{AlexRGB}
  Alex Teichman, Sebastian Thrun,
  Learning to Segment and Track in RGBD.

\bibitem{AlexDepth}
  Alex Teichman, Sebastian Thrun,
  Tracking-Based Semi-Supervised Learning.

\bibitem{SSVM9}
T. Joachims, T. Finley, and C.-N. Yu. Cutting-plane training of structural svms. Machine
Learning, 77(1):2759, 2009.

\bibitem{SSVM20}
M. Szummer, P. Kohli, and D. Hoiem. Learning crfs using graph cuts. In ECCV, 2008.

\bibitem{SSVM21}
21. B. Taskar, V. Chatalbashev, D. Koller, and C. Guestrin. Learning structured prediction models:
A large margin approach. In ICML, 2005.

\end{thebibliography}

\bibliography{egbib}
}

\end{document}
